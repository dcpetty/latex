%%%%%%%%%%%%%%%%%%%%%%%%%%%%%% Macros
\newcommand{\code}[1]{\texttt{#1}}
\newcommand{\lemail}[1]{\texttt{\textless\href{mailto:#1}{#1}\textgreater}}
\newcommand{\prob}[1]{\vspace{4pt}\noindent\textbf{\normalsize{#1}}\\}
\newcommand{\statement}[1]{\emph{#1}}
\newcommand{\divider}{\vskip4pt\hrule\vskip2pt}
\newcommand{\hwnote}[1]{\begin{tcolorbox}[colback=gray!5!white,colframe=BrickRed,title=\textsc{\textbf{Note}}]\textcolor{BrickRed}{#1}\end{tcolorbox}}
\newtcolorbox{tcbquote}[2][]{%
    colback=gray!5!white,
    grow to right by=-1em,
    grow to left by=-1em, 
    boxrule=0pt,
    boxsep=0pt,
    breakable,
    enhanced jigsaw,
    borderline west={2pt}{0pt}{gray},
    title={#2\par},
    colbacktitle={gray},
    coltitle={black},
    fonttitle={\bfseries},
    attach title to upper={},
    #1,
}
\definecolor{rebeccapurple}{rgb}{0.4, 0.2, 0.6}
\newtcolorbox{sidebarquote}[2][]{%
  enhanced,
  boxrule=0pt,
  colback=rebeccapurple!5!white, % Light purple background
  breakable,
  sharp corners,
  borderline west={2pt}{0pt}{rebeccapurple!80!black}, % Thick left border
  colbacktitle=rebeccapurple!80!black, % Optional: title bar
  fonttitle=\bfseries\upshape,
  #1
}
\def\foreign{\em}
\def\etc{{\foreign etc.}}
\def\eg{{\foreign e.g.}}
\def\ie{{\foreign i.e.}}

%%%%%%%%%%%%%%%%%%%%%%%%%%%%%% Math
\newcommand*{\dx}{\,dx}
\newcommand*{\bb}[1]{\mathbb{#1}}
\newcommand*{\N}{\bb{N}}
\newcommand*{\primes}{\bb{P}}
\newcommand*{\Q}{\bb{Q}}
\newcommand*{\R}{\bb{R}}
\newcommand*{\U}{\bb{U}}
\newcommand*{\Z}{\bb{Z}}
\newcommand*{\RZ}{\bb{R}_{\ge0}}
\DeclareMathOperator{\rref}{RREF}
\DeclareMathOperator{\proj}{proj}
\DeclareMathOperator{\refl}{ref}
\DeclareMathOperator{\imag}{im}
\DeclareMathOperator{\spn}{span}
\DeclareMathOperator{\rank}{rank}
\DeclareMathOperator{\tr}{tr}
\DeclareMathOperator{\adj}{adj}
\DeclareMathOperator{\cm}{comatrix}
\DeclareMathOperator{\minor}{minor}

\newcommand*\xor{\oplus}

%https://tex.stackexchange.com/questions/235118/making-a-thicker-cdot-for-dot-product-that-is-thinner-than-bullet
\makeatletter
\newcommand*\adot{\mathpalette\adot@{.5}}
\newcommand*\adot@[2]{\mathbin{\vcenter{\hbox{\scalebox{#2}{$\m@th#1\bullet$}}}}}
\makeatother

% https://tex.stackexchange.com/questions/33519/vertical-line-in-matrix-using-latexit
\makeatletter
\renewcommand*\env@matrix[1][*\c@MaxMatrixCols c]{%
  \hskip -\arraycolsep
  \let\@ifnextchar\new@ifnextchar
  \array{#1}}
\makeatother
% https://tex.stackexchange.com/questions/295293/vec-command-produces-tilde-instead-of-arrow
\def\vec{\mathaccent "017E\relax }
% https://tex.stackexchange.com/questions/199789/which-bold-style-is-recommended-for-matrix-notation
\newcommand{\mat}[1]{\mathbf{#1}}
\newcommand{\matinv}[1]{\mathbf{#1}^{-1}}
\newcommand{\trans}[1]{\xrightarrow{\text{#1}}}
\newcommand{\Trans}[2]{\xrightarrow[#2]{\text{#1}}}
\newcommand{\vecn}[2]{\vec{\ifx#2\empty#1\else#1_{#2}\fi}}

% https://tex.stackexchange.com/questions/33401/a-version-of-colorbox-that-works-inside-math-environments
% \highlight[<colour>]{<stuff>}
\newcommand{\highlight}[2][yellow]{\mathchoice%
  {\colorbox{#1}{$\displaystyle#2$}}%
  {\colorbox{#1}{$\textstyle#2$}}%
  {\colorbox{#1}{$\scriptstyle#2$}}%
  {\colorbox{#1}{$\scriptscriptstyle#2$}}}%

% https://tex.stackexchange.com/questions/3033/forcing-linebreaks-in-url
\def\UrlOrds{\do\*\do\~\do\'\do\"\do\-\do\_}%
\makeatletter
\g@addto@macro{\UrlBreaks}{\UrlOrds}
\makeatother

%%%%%%%%%%%%%%%%%%%%%%%%%%%%%% Linear Algebra
\newcommand{\RR}{\R^{2}}
\newcommand{\RRR}{\R^{3}}
\newcommand{\RRN}{\R^{n}}

% Matrices for A, B, C, In
\newcommand{\matA}{\mat{A}}
\newcommand{\matB}{\mat{B}}
\newcommand{\matC}{\mat{C}}
\newcommand{\matD}{\mat{D}}
\newcommand{\matI}{\mat{I}}
\newcommand{\matIn}{\mat{I_{n}}}
\newcommand{\matQ}{\mat{Q}}
\newcommand{\matR}{\mat{R}}
\newcommand{\matS}{\mat{S}}
\newcommand{\matSinv}{\matinv{S}}

% Vectors for e, u, v, w, x, y, z
\newcommand{\vecen}[1]{\vecn{e}{#1}}
\newcommand{\vece}{\vecen{}}
\newcommand{\vecun}[1]{\vecn{u}{#1}}
\newcommand{\vecu}{\vecun{}}
\newcommand{\vecvn}[1]{\vecn{v}{#1}}
\newcommand{\vecv}{\vecvn{}}
\newcommand{\vecwn}[1]{\vecn{w}{#1}}
\newcommand{\vecw}{\vecwn{}}
\newcommand{\vecxn}[1]{\vecn{x}{#1}}
\newcommand{\vecx}{\vecxn{}}
\newcommand{\vecyn}[1]{\vecn{y}{#1}}
\newcommand{\vecy}{\vecyn{}}
\newcommand{\veczn}[1]{\vecn{z}{#1}}
\newcommand{\vecz}{\veczn{}}

% Bases for B, E, U
\newcommand{\basisB}{\mathcal{B}}
\newcommand{\basisE}{\mathcal{E}}
\newcommand{\basisU}{\mathcal{U}}

% Eigenvalues
\DeclareMathOperator{\am}{AM}
\DeclareMathOperator{\gm}{GM}
\newcommand{\lIA}{\lambda\matI-\matA}
\newcommand{\kerlIA}{\ker\left(\lIA\right)}

%%%%%%%%%%%%%%%%%%%%%%%%%%%%%% Classes
\def\apcs{{\bf AP Computer Science}}
\def\apcsp{{\bf AP Computer Science Principles}}
\def\ga{{\bf Geometry Advanced}}
\def\ig{{\bf Informal Geometry}}
\def\ia{{\bf Integrated Analysis}}
\def\pcone{{\bf Precalculus Part 1}}
