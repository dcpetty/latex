\documentclass[11pt,twocolumn]{article}%
\usepackage{amsmath,amsfonts,amssymb,enumitem,fancyhdr,hyperref,verbatim}
\usepackage[paper=letterpaper, margin=3cm]{geometry}
\usepackage{palatino}

\def\foreign{\em}
\def\etc{{\foreign etc.}}
\def\eg{{\foreign e.g.}}
\def\cmtw{{\em Classroom Management That Works}}
\def\cmtwfull{\cmtw: {\em Practical Solutions to
Persistent Problems\texttrademark}}

\fancyhf{} % clear all header and footer fields
\fancyhead[L]{\bf \leftmark}%
\fancyhead[R]{\bf \thepage} %
\fancyfoot[L]{\scriptsize \bf David C. Petty\\POBox 400222
\textbullet \ Cambridge MA 02140-0003 USA \textbullet \
\texttt{\textless dcp@mail.com\textgreater} \textbullet \
+1.617.331.2345}%
\fancyfoot[R]{\scriptsize 2005/07/30\\\jobname.pdf}%
\renewcommand{\headrulewidth}{0.5pt}
\renewcommand{\footrulewidth}{0.5pt}
\pagestyle{fancy}%

\setlength{\columnsep}{1cm}%
\setlength{\parindent}{0cm}%
\setlength{\parskip}{0.5cm}%
\setlist{nolistsep, itemsep=.125\baselineskip}%
\setitemize{leftmargin=*, label={\tiny$\blacktriangleright$}}%
\setdescription{style=multiline, leftmargin=3cm}%

\usepackage{natbib}% for \bibliographystyle{agsm}
\bibliographystyle{alpha}

\begin{document}

%%%%%%%%%%%%%%%%%%%%%%%%%%%%%% Title
\title{\large {\bf Classroom Management That Works: Practical
Solutions to Persistent Problems\texttrademark}}%

%%%%%%%%%%%%%%%%%%%%%%%%%%%%%% Author(s) (including \thanks)
\author{David C. Petty %
\thanks{The author wishes to thank the teacher and fellow students in
\href{http://connectinglink.com}{The Connecting Link} \cmtwfull\
class held during week 30 of 2005.} \\ %
{\small \texttt{\textless dcp@mail.com\textgreater} \textbullet \ +1.617.331.2345}}%

%%%%%%%%%%%%%%%%%%%%%%%%%%%%%% Date
\date{ } %

\maketitle

\begin{abstract}

%%%%%%%%%%%%%%%%%%%%%%%%%%%%%% Keywords & Abstract

{\em Keywords: teaching, classroom management, discipline %

This paper discusses internal and external aspects of classroom
management and personal action plans for improving the ability to
(a) always assume the best about students (an internal aspect) and
(b) give students choices in what they do in the classroom (an
external aspect). This paper also discusses five classroom
management challenges experienced in the first year teaching middle
school mathematics and appropriate responses to those challenges.

}

\end{abstract}

\baselineskip = 1.3\baselineskip

%%%%%%%%%%%%%%%%%%%%%%%%%%%%%% Introduction
\section{Introduction}
\label{Introduction}

%\ref{Introduction}

This paper is in fulfillment of requirements for a
\href{http://connectinglink.com}{The Connecting Link} class titled
\cmtwfull\ held during week 30 of 2005 and for which three graduate
credits are to be awarded by \href{http://salemstate.edu/}{Salem
State College}.

In this paper I discuss internal and external aspects of classroom
management and personal action plans I developed for improving my
ability to (a) always assume the best about my students (an internal
aspect of my skill set) and (b) give students choices in what they
do in the classroom (an external aspect of my skill set). In this
paper I also discuss five specific classroom management challenges I
experienced in my first year teaching middle school mathematics and
what I believe are appropriate responses to those challenges for use
in my next classroom.

%%%%%%%%%%%%%%%%%%%%%%%%%%%%%% Classroom Discipline
\section{Classroom Discipline}
\label{Classroom Discipline}

There are many aspects to classroom discipline and the classroom
management that maintains it. Teaching curriculum content to a class
can only be successful when students are motivated to learn and
there is a classroom environment conducive to learning.

All parents know that two motivational tools for children are {\em
threats} and {\em bribery}. Those are {\em extrinsic} motivations
--- consequences and rewards come from the outside. Student motivation
in the classroom is best, however, when it is {\em intrinsic} ---
sometimes referred to as being 'self-motivated' --- and comes from
within. Intrinsic motivation for students is an analog of internal
teacher discipline discussed in section \ref{Internal Discipline}.

A smoothly running classroom will be the difference between students
learning a lot and students learning a little. Classroom environment
is determine by external factors set by the teacher. A classroom
environment conducive to learning is an analog of external teacher
discipline discussed in section \ref{External Discipline}.

\subsection{Internal Discipline}
\label{Internal Discipline}

Internal discipline for a teacher comes from an inner authority and
the ability to invisibly express it to students. The word {\em
authority} comes from {\foreign auctor} (the Latin word for {\em
author}), which literally means `one who causes to grow.' Classes
accept teachers' authority to the extent that teachers are able to
be the author of what happens in their classrooms.

Examples of things that teachers can author in their classrooms
include:
\begin{itemize}
 \item a vision of an exciting year of learning;
 \item a safe environment --- physically, emotionally, and intellectually ---
 for students to participate, ask questions, work together, and share their
 work;
 \item a caring relationship and positive connections to every student;
 \item immediate productive feedback.
\end{itemize}

Teachers with authority exude leadership that comes from within. In
my first year of teaching I had authority in areas like curriculum
content and guiding class participation, but I was lacking in some
aspects of inner authority. One thing that was difficult for me to
consistently maintain is {\em always assuming the best of students}.

When I began teaching, like all first-year teachers, I was thrust
into a classroom in front of students for whom I was expected to
deliver scintillating content in a nurturing environment. That was
good as far as it went, but as the year progressed there were
certain students whose job (it seemed) was to get the class off
task, to distract other students, and to generally do as little as
possible. When this happened day after day, I found myself coming to
class expecting this behavior from those students.

\subsubsection{Assume the Best}
\label{Assume the Best}

When I started the year I had high expectations for every student, I
believed in every student, and I thought the best of every student.
The grading policy document I sent home to families at the beginning
of the school year stated (in part):

\begin{quote}\small
I believe in every student and I expect every student to learn
mathematics.
\end{quote}

One of the most valuable touch stones I took away from \cmtw\ is
`think / feel / know,' or `head / heart / gut.' My interpretation of
that mnemonic is that we can over-intellectualize with our rational
thoughts and our emotions --- and that sometimes we just have to `go
with the gut.' That was especially difficult for me during the
school year when I was either overwhelmed with thoughts and emotions
or so busy in the class that I failed to have perspective.

When I failed to assume the best of students that challenged my head
and heart day after day, I lost the perspective of my gut that told
me that every student wants to learn content and every student wants
to learn behavior. In \cmtw\ I learned that assuming the best of
every student every day is the {\em basis} of conscious classroom
management \cite{book:ccm}, rather than something impossibly beyond my
grasp.

\subsubsection{Improvement Action Plan -- Internal}
\label{Improvement Action Plan -- Internal}

When asked what is good about teaching, I can answer about what I
{\em think} about teaching (`I am providing valuable content to
students in a gatekeeper subject that will be vital to them if they
want to be scientists or engineers\dots') or what I {\em feel} about
teaching (`I am making a difference in the lives of students and
giving them confidence in a popular culture that does its best to
tear them down and lie to them\dots'), but what I {\em know} about
teaching is that all the intellectual reasons come together to
inform my gut that tells me teaching is the right thing for me to
do.

The basis for my action plan is to `go with the gut' and assume the
best of every student every day, because that is what I felt at the
beginning of the school year and anytime I was able to step back
from the distracting thoughts and feelings, take a deep breath, and
get some perspective on my students. My action plan is as follows:

\begin{itemize}

\item {\bf Visualize} --- It is routine for athletes to visualize the
outcome of a competition before it starts. Visualization is
sometimes called `guided imagery' and it involves crating visual,
kinesthetic, or auditory images of a desired outcome in one's mind.
I will visualize success for my students.

\item {\bf Personalize} --- I discussed maintaining perspective with
a colleague who has 30 years experience teaching elementary school.
She works hard each year to be able to identify to herself something
that every student in the class is the best at --- whether it's
asking questions about science, or lining up when leaving the room,
or perhaps just being mischievous. She uses those thoughts to
remember to assume the best of students during those times when that
is the most difficult. She also said that, though it might take her
most of the year to come up with what certain students are best at,
only twice in 30 years did she fail to come up with {\em something}
before the end of the year. I will look for at least one thing that
every one of my students is the best at.

\item {\bf Response-ability} --- When given a choice by circumstance
or by student behavior (and in the face of overwhelming rational
thoughts or emotions) I will base my responses on the assumption
that my students want to learn. \footnote{This is consistent with at
least two of {\em The 7 Habits of Highly Effective People}
\cite{book:7habits}.}

\end{itemize}

\subsection{External Discipline}
\label{External Discipline}

External discipline for a teacher is most noticeable in the
day-to-day running of the classroom. Examples of things that
manifest a teacher's external discipline include:

\begin{itemize}

\item students remaining on task;
\item momentum maintained in lessons, class discussions, and seat and group work;
\item the level of disruptiveness;
\item the level of engagement and participation;
\item the degree to which students have learned {\em procedures}.

\end{itemize}

Teachers with smoothly running classrooms have spent significant
effort teaching procedures along with content. Smith
\cite[p.~90]{book:ccm} recommends {\em two procedures per lesson}. in
\cmtw\ I learned many external techniques that contribute positively
to classroom environment. An area that I was conscious of in my
first year teaching and that I {\em thought} I had covered, but in
retrospect I realize that I was not always successful at, is to give
students {\em real} choices.

% classroom agreements, activity procedures, directions, transition procedures.

\subsubsection{Give Real Choices}
\label{Give Real Choices}

When I started my first year teaching I was committed to getting
students involved in every phase of learning and classroom activity.
My commitment was genuine and my execution was hopeful. As the year
wore on, however, I learned that involving students in choices could
also be a `can of worms' that could take a lot of class time without
great benefit. My inexperience led me to waffle over which choices
were important and fruitful and which were superfluous.

That consequently led me to reduce the number of classroom
opportunities for student choice and when I did give the students
choices, they were sometimes not {\em real} choices. For certain
lesson, classroom procedure, or safety reasons a
because-I'm-the-teacher (non)-choice is OK. I learned from \cmtw\
that students {\em need} choices and, therefore, student choices
must not be {\em non}-choices\footnote{A Tom Hermansen illustration
\cite[p. 65]{book:ccm} has the caption, ``\dots you can go to the
principal's office {\bf now}\dots or you can go to the principal's
office {\bf immediately}.''}.

\subsubsection{Improvement Action Plan -- External}
\label{Improvement Action Plan -- External}

The basis for my action plan is to give students {\em real} choices
every day in my classroom. My action plan is as follows:

\begin{itemize}

\item {\bf Involvement} --- The first aspect of student choice is
having students make choices. From my point of view as the teacher,
that means being conscious of, and finding the balance between, when
choices are inappropriate or irrelevant and when real choice will
enhance student involvement -- individual choice of work, class-wide
choices of sequence, choices between two equally valid outcomes,
\etc. Once I have chosen the appropriate opportunity for student
choice, I must then insure that there are real and appropriate
choices for students to make -- that involves teaching as a
procedure what each choice entails and what the expected outcomes
are. Then I need a mechanism to insure that the classroom is
appropriately balanced -- so that there aren't too many students
with any one choice and students aren't allowed to repeatedly make
the same choice.

\item {\bf Reiteration} --- Unless choices are made for a `one shot
deal,' students must be reminded of the choices they made and (when
appropriate) be allowed to revisit those choices. That is just one
aspect of `polishing the railroad tracks' \cite[p. 80]{book:ccm} be
reiterating the choices that the students have made. I was mistaken
in my first year teaching in thinking that once we had agreed on a
class choice and hung a sign on the wall reflecting that choice,
then the matter was forever settled. {\em Reiterating and revisiting
choices are crucial}.

\item {\bf Follow through} --- Choices are not choices if they have
no effect. As part of teaching the procedure embodied in the choice,
students need follow through on behalf of the teacher and the other
students. Whether it is codifying the choices and hanging them on
the wall, or using the choices in a classroom procedure day after
day, or enforcing student choice within a classroom or laboratory, I
must consistently remember the choices made by the students and
remind the students of them. If choices are allowed to change
arbitrarily or are negligently reversed by students or teachers,
then students will (rightly) not believe in them.

\end{itemize}

%%%%%%%%%%%%%%%%%%%%%%%%%%%%%% Classroom Management Problems
\section{Classroom Management}
\label{Classroom Management}

In section \ref{Classroom Management} I describe five classroom
management situations that posed difficulties for me in my first
year of teaching and that I felt ill-equipped to handle. In each
situation I describe, based on what I have learned in \cmtw, how I
will handle it next year.

As with any classroom technique (content or behavior), seeing is
believing. There were many techniques I used in my first year that
surprised me in their effectiveness. I am anxious to put the
techniques I learned into practice to see how and when they work.
Only then will I have the inner authority and confidence to be a
leader in my classroom.

\subsection{Soft Voice Versus Loud Voice}
\label{Soft Voice Versus Loud Voice}

Yelling at the class. Every teacher has done it. Every teacher has
experienced it in their own schooling. It comes from losing control
of the noise level in the class or getting angry at the class (or
both!). When I yelled at my class, I was getting their attention by
cutting through the noise level and I was expressing frustration
with how little learning was going on. Yelling would encourage any
students whose out-of-school situation or natural tendencies allowed
them to yell back to escalate the situation. I always found that
yelling would wrest temporary control of the class back to me, but
that the effect was never lasting.

Smith describes the {\em firm and soft paradox} \cite{book:ccm}.

\begin{itemize}\item Your voice goes down in volume. \item Your voice
goes lower in tone. \item Your body squarely faces the
student.\end{itemize}

In my experience it is right to call it a paradox, because the
response is the opposite of the natural tendency --- to raise your
voice to get the attention of the class --- when dealing with
disruption. This approach will not work if implemented in isolation
so that your voice gets softer and lower as the class's gets louder
and higher. This technique is meant to be combined with others that
teach the class procedures for working together and used to
preemptively address individual student behavior by simultaneously
being firm in my resolve and deescalating the conflict.

I will treat students who are resistent by assuming the best of them
and with the patience and firmness --- including using an effective
`no' --- to resolve the behavior before it escalates. This is very
different from what I sometimes did in my class in my first year
when I gave up too soon and did not deal with individual student
disruptions completely or raised my voice. In either case, the
results were generally disastrous.

\subsection{Unpreparedness Disrupts Class}
\label{Unpreparedness Disrupts Class}

There were days in my class where it seemed that everyone came to
class unprepared without books, pencils, or homework. Invariably
that would lead to disruption at the beginning of class or during
transitions to seat or group work --- including fumbling with
notebooks, walking around class to sharpen pencils, and requests to
go to lockers. I was confused by this behavior, because I (and
several other teachers on the middle school team) had made
preparedness part of the grade and every student knew about it from
the beginning of the school year. Students with organizational
learning disabilities or difficult family situations would
occasionally come to class with nothing and I also had to
accommodate their needs.

The emphasis in \cmtw\ on teaching procedures was a revelation to
me. I was not prepared for how much {\em teaching} has to go into
procedures, how much {\em practice} students need to master them,
how much {\em reinforcement} they need to remember them, and how
much they {\em want} to learn them. Understanding and acting on that
reality will be a major difference for me between my first year and
my next year teaching.

I can see now that students had not internalized the behavior for
being prepared and that I did not treat the start of each class in a
consistent manner (\eg during attendance) to help them to be
prepared. At the end of the school year, before and after MCAS, the
class had a warm-up assignment on the overhead every day and that
helped to focus class attention on getting ready. Next year I will
have a more consistent start-of-class procedure to insure that every
student is prepared before the lesson starts and I will spend some
time every day teaching and reinforcing that procedure.

\subsection{Refusing to Work in Groups}
\label{Refusing to Work in Groups}

It was common in my class for students to work in groups. There were
always some students who claimed that they could not work with one
another. Though we spent time at the beginning of the school year
working on norms for working together and procedures for group work,
there was invariably complaining when I assigned new groups ---
sometimes to the point of keeping the rest of the class from
working. It was also typical that, during the first session or two,
that the dysfunctional groups would have one or more students
refusing to participate. Some students with learning disabilities
were singled out as `I can't work with him!.'

The emphasis in \cmtw\ on teaching procedures was a revelation to
me. I was not prepared for how much {\em teaching} has to go into
procedures, how much {\em practice} students need to master them,
how much {\em reinforcement} they need to remember them, and how
much they {\em want} to learn them. Understanding and acting on that
reality will be a major difference for me between my first year and
my next year teaching.

I have learned in this course that my biggest mistake was in
assuming that once we developed the norms for group work, and I
taught the group-work procedure once, and the students followed the
group-work procedure once, that we were finished. The group-work
procedure is complicated (moving desks, collecting materials,
assuming roles, following directions, working with difficult
partners, \etc) and must be taught repeatedly --- effectively
polishing the railroad tracks of procedure so the train of content
can run smoothly over them \cite[p. 80]{book:ccm}. I will make sure that
group-work procedures are thoroughly taught.

\subsection{Comments Lead to Conflict}
\label{Comments Lead to Conflict}

Students --- especially middle-school students --- can be cruel. My
school is one that values and works diligently on school culture and
there is zero tolerance within that culture for bullying. Though we
don't have a formal anti-bullying program, from the principal on
down to the kindergarten, words like `caring' and `respect' have
real meaning and are constantly discussed and reinforced by
teachers. {\em In spite} of that, adolescents can make inappropriate
comments to one another in class. A big problem I had in my first
year of teaching was diffusing the shouting match that could break
out over inappropriate comments and their retaliations ---
especially when the original comments were often whispered or
mumbled so as to be inaudible to me.

In \cmtw\ I learned to assume the best of students about their
desire to learn, that a smoothly running classroom environment is
what {\em students} (not just teachers) prefer, and that dealing
with problems preemptively is the hallmark of an authoritative
classroom leader. I believe that improving my performance in all
those areas will minimize the opportunities for students to focus on
one another rather than the work.

It is important to me, however, to not let such inappropriate or
cruel comments slide. One of the things I have learned from in this
class is {\em procedure}, {\em procedure}, {\em procedure} (to
paraphrase a truism from real estate). Since I believe class norms
are important and if I have given students the opportunity to help
chose the most important norms, then I must periodically teach them
as a procedure if I expect the students to keep them internalized.

\subsection{Inveterate Argumentativeness}
\label{Inveterate Argumentativeness}

I told the parent of one of my seventh-grade students that, if she
ever needed a recommendation to law school, that she could count on
me! That student took argumentativeness to a new extreme. She is
extremely smart and logical and winning an argument with her was
impossible. I know, because I tried. It's not that didn't have
better arguments or more authority, but because I engaged in the
argument at all allowed her to switch the focus to another argument
before the first one was resolved, thereby prolonging the
interaction and amplifying its disruption. The cycle of
argumentation is one that can be made worse, if engaged in, by
students with attention deficits, or emotional needs for the
spotlight.

Though I thought I was practicing the `effective no' and `standing
my ground,' I was not. There were several ways that my `no' was not
effective:

\begin{itemize}\item It was not simply `no.' \item I over-explained
by engaging in argument. \item I allowed wiggle room. \item I did
not effectively employ the `broken record' technique. \item I was
not patient.
\end{itemize}

Next year I will stand my ground and use an effective `no' and I am
thankful for the `trial by fire' of argumentativeness afforded me by
that student that gave me the opportunity to learn.

%%%%%%%%%%%%%%%%%%%%%%%%%%%%%% Conclusion
\section{Conclusion}
\label{Conclusion}

\href{http://connectinglink.com}{The Connecting Link} \cmtw\ class
was full of insight on how to bring leadership and authority
--- {\foreign in lieu} of simply management --- to a classroom. The
class materials (including \cite{book:ccm}, \cite{book:rcm}, and the
participant manual) provided a wealth of techniques, examples, and
exercises.

My teacher was effective at presenting material and guiding
classroom activities and discussions. My colleagues were fun to work
with, thoughtful, and (above all) helpful. They were also patient
with my many first-year-teacher questions.

I will be able to bring many things I learned in \cmtw\ into my
classroom in the fall.

\bibliography{../../bib/dcpbibtex} % To update, delete *.blb

\end{document}
