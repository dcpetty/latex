\documentclass[12pt]{article}%
%%%%%%%%%%%%%%%%%%%%%%%%%%%%%%%%%%% Packages
\input ../../input/input-packages.tex
\geometry{margin=1in,top=0.75in,bottom=0.75in,head=0in}%
\usepackage{dblfnote}%

%%%%%%%%%%%%%%%%%%%%%%%%%%%%%%%%%%% Document data
\input ../../input/input-data.tex

%%%%%%%%%%%%%%%%%%%%%%%%%%%%%% Macros
\input ../../input/input-macros.tex

%%%%%%%%%%%%%%%%%%%%%%%%%%%%%% Headers & footers
%\input ../../input/input-header-footer.tex

%%%%%%%%%%%%%%%%%%%%%%%%%%%%%% Document style
%\input ../../input/input-style.tex

%%%%%%%%%%%%%%%%%%%%%%%%%%%%%% Table of contents / index / bibliography
\input ../../input/input-toc-idx-bib.tex

%%%%%%%%%%%%%%%%%%%%%%%%%%%%%% Points and solutions
\input ../../input/exam-points-solutions.tex

%%%%%%%%%%%%%%%%%%%%%%%%%%%%%% macros
\def\foreign{\em}
\def\etc{{\foreign etc.}}
\def\eg{{\foreign e.g.}}

%%%%%%%%%%%%%%%%%%%%%%%%%%%%%% document
\begin{document}

\begin{center}{\large\bf How Fo' Speak Pidgin}\end{center}

%%%%%%%%%%%%%%%%%%%%%%%%%%%%%%%%%%% memo body

This is a transcript of the \href{http://youtube.com/}{YouTube} video titled {\em\href{http://youtu.be/GLmfQSR3EI0}{How Fo' Speak Pidgin}} posted by {\em masterhide} in 2010. It has garnered > 500,000 views. For best results, follow along with this document while listening. Please forward any and all corrections to \href{mailto:pidgin@patton-petty.net}{me}. Thanks.

\vspace{5mm}\hrule\vspace{5mm}

'K las' week my English teacher wen tell us our assignment due dis week is fo' write one paper, eh? An' make em on somet'ing creative. So I neva know what fo' write abou' an' I was tryin' write abou' all dis random stuff. Like how fo' write dis, how fo' do dat, like how fo' be li' dis, an' den I was like, shoots I jus' goin' write one paper how fo' speak Pidgin.

Den I started proof reading li' dat. You know, you get da rough draf' an' started proof reading an' I was like: no way she gonna understan' dis. She jus goin' t'ink, ho, dis girl, she don' know how fo' spell, yeah? She don' know how fo' write. 

So, I t'ought maybe I go turn in da paper, den I go make one video, put em on da internet, li' dat. An' den maybe when she hear em talk... when she hear me speak em, den she goin' understan', eh? Maybe if da uddah kids in class hear em, maybe dey understan'. (Maybe dey jus' hate me even more, I don' know. It's like opposite kill-haole day, yeah, dey kill a Hawaiian, but.) ANYWAY...

So, I got my notes right here on da computer screen. So sometime I not goin' look at you, 'k? I goin' look at da screen, 'cause I know, I be gettin' old, 'k? I t'ink forget, but.

I goin' read you my paper on {\em How Fo' Speak Pidgin}. So, here we go. Ready? You ready? I ready. I don' know if I ready. I gotta drink some water firs', eh? Is it water? Who knows. Only I know. You don' know. Anyway.

Dis paper called {\em How Fo' Speak Pidgin}. By me. An', begin. So, mau kau kau?\footnote{mau kau kau? = is everyone ready?} Anyway.

\vspace{5mm}
\hrule
%\vspace{5mm}

\begin{multicols}{2}
In dis paper I goin' teach you how fo' speak Pidgin. But it stay one of dose t'ings dat mo bettah I show you. Maybe you askin' yourself, how come peopo talk li' dat? Why dey no speak good like us? An' fo' real, it not like we no can. If can, can. But mos' a time mo bettah meaning come from speaking Pidgin. Choke\footnote{choke = a lot} get, you know, {\em lost in translation} li' dat.

Firs' maybe you need fo' learn some a da history surrounding dis language dey now calling ``Hawaiian Creole.'' See da buggah started in Hawai`i, no joke. But da t'ing get so many uddah causes it get ha'd fo' count, eh? Get da Japanese, da Chinese, da Portuguese, an', ho, no fo'get da Filipinos, 'k? Pinay\footnote{\href{http://en.wikipedia.org/wiki/Pinoy}{Pinay} = Filipina (or \href{http://globalnation.inquirer.net/80871/why-filipino-americans-say-pilipino-not-filipino/}{Pilipina})}!

Why Aunty? Why come so many, you ask? Cause, my keiki\footnote{keiki = child}, way back when, afta our Queen Liliuokalani weh haf' fo' leave da palace 'cause da missionaries wen do dat [attegration]\footnote{no idea what this word is... can anyone help?} stuff, afta dey wen give everyone permanent shaka from da leprosy li' dat an' send em to da island, you know, dey need mo way fo' make money, eh? An' you know dose guys, dey wen make choke pilikia\footnote{pilikia = trouble} fo' everybody. So dey wen go all ova, take all kine peopo from everywhere. Make em work da plantation in da hot sun, picking pineapple, burning sugar cane, you know somet'ing like fo' do crooks, eh? Yeah, no worry, baby, Aunty t'ink so too. 

No aloha... no ha, dose ones. An' ha, breath of life, eh? An' ole means no. Dats why we call dose kines wit no aloha ``haole.'' No breath. You starting fo' understan', baby?

So all dese peopo come toget'a, trying fo' making work, trying fo' be happy, trying fo' work ha'd, but guess what? No can talk! Nobody can talk to each udda. 'Cause nobody speak da same language, eh? So dey sta't talking wit dem guys who no undastan' call ``broken English.'' But it mo dan jus' dat. It not like we jus' broke da English, 'k? Get plen'y words from udda language. 

Fo' example, you going say, ``Oh, Timmy, would you please be a good boy, and take a bath, and go to bed, son?'' Or how abou', ``Eh, boy, go `au `au\footnote{`au `au = bathe} moe\footnote{moe = sleep} already!'' Which one your boy goin' lis'en to? I don' know abou' you, but my kid's hot head. I gotta ac' like I goin' lick em fo' make em lis'en. An' when you use udda words li' dat, an' no translate, you can see em mo' fas', mo' bettah efficiency (maybe from da Japanese li' dat, I don' know). But it's mo bettah.

[...Long, long time ago in da aina\footnote{aina = land}.]

Growing up da way we wen, no mo' time fo' talk when your okole\footnote{okole = butt} in trouble. I mean you can talk story\footnote{talk story = chat} and kanikapila\footnote{kanikapila = jam session} all kine when you pau\footnote{pau = finished} hana\footnote{hana = work}, but in da moment when da teacher looking at you, and da boss like fire you, no can do not'ing, what you goin' do? Try give one explanation? Dey no going' lis'en to one local person trying fo' talk like dem. You jus' goin' say some fas' words and blame em on da kine\footnote{da kine = da kine}. 'Cause you know what da kine is, but dey don' know what da kine is. You know da kine? So it's good, you know. Dey don' know what you talking abou' and let you live one udda day to go pick pineapple an' get all kine centepede bites go pio\footnote{pio = out of sight} on your okole coming back from wen you pau hana looking like you wen huli maka flip\footnote{huli maka flip = `ass over teakettle'}!

Anyway, Aunty getting off da subject, but da poin' is: Pidgin is not jus' fo' dummies, 'k? Like all you peopo in da mainland t'ink it is. It stay one language strong in mana\footnote{mana = spiritual energy}, get choke aloha, an' deep family roots. It bring us closer toget'a an' make us understan' each udda mo bettah, eh? An' isn't dat da meaning of da kine, you know, life, eh? To love one an udda, understan' each udda, get along. 

So I guess Aunty neva teach you not'in' abou' how fo' speak. Maybe Aunty wen at leas' teach you why we talk like dis. And dats da firs' step fo' learning, right? Is knowing how come. 

So you still like learn how fo' talk? Well, come I take care you e komo mai\footnote{e komo mai = welcome}. Da only way for you to learn is for you to come to Hawai`i and experience waha\footnote{waha = talk} firs' hand, you know. You like learn? I can try, but no can tell you. Mo bettah I show you.

\end{multicols}

%\vspace{5mm}
\hrule
\vspace{5mm}

Not ``shoyu,'' ``show you.'' Soy sauce, what? Loco moco\footnote{loco moco = rice, hamburger, egg, \& gravy}, 'k. Mahalo\footnote{mahalo = thank you}!

\end{document}

