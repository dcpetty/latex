\documentclass[10pt]{article}%
%%%%%%%%%%%%%%%%%%%%%%%%%%%%%%%%%%% Packages
\input ../../input/input-packages.tex

%%%%%%%%%%%%%%%%%%%%%%%%%%%%%%%%%%% Document data
\input ../../input/input-data.tex

\def\dochomework{}
\def\hw{Notes on equations of motion} % Change document homework header
\def\docdate{2020/10/10} % Change document date
\def\docphone{+1.772.444.2710} % Change document phone

%%%%%%%%%%%%%%%%%%%%%%%%%%%%%% Macros
\input ../../input/input-macros.tex

%%%%%%%%%%%%%%%%%%%%%%%%%%%%%% Headers & footers
\input ../../input/input-header-footer.tex

%%%%%%%%%%%%%%%%%%%%%%%%%%%%%% Document style
\input ../../input/input-style.tex

%%%%%%%%%%%%%%%%%%%%%%%%%%%%%% Table of contents / index / bibliography
\input ../../input/input-toc-idx-bib.tex

%%%%%%%%%%%%%%%%%%%%%%%%%%%%%% Points and solutions
\input ../../input/exam-points-solutions.tex
%\printanswers% to print solutions

\begin{document}
\setlength{\baselineskip}{1.2\baselineskip}%
\begin{multicols*}{2}

%%%%%%%%%%%%%%%%%%%%%%%%%%%%%%%%%%%%%%%%%%%%%%%%%%%%%%%%%%%%%%%%%%%%%%
\prob{Differential equations}
\statement{\href{https://en.wikipedia.org/wiki/Equations_of_motion\#Constant_translational_acceleration_in_a_straight_line}{Equations of motion} with constant acceleration are based on the differential equations: $v(t) = \frac{d\,x(t)}{dt}$, $a(t) = \frac{d\,v(t)}{dt}$. Integrating for $a(t) = \text{constant}$:}
\begin{align}
 a(t) &= \text{constant} \\
 v(t) &= \int a \, dt =  \boxed{at  + v_{0}} \\
 x(t) &= \int v \, dt = \boxed{\frac{1}{2}at^{2} + v_{0}t + x_{0}}
\end{align}
For any interval $\left[ t_{1}, t_{2} \right]$ and for $\Delta t = \left( {t_{2}} - {t_{1}} \right)$...
\begin{align}
 x(t)\bigg\rvert_{t_{1}}^{t_{2}} = \int_{t_{1}}^{t_{2}} v \, dt &= \left( \frac{1}{2}a{t_{2}}^{2} + v_{0}{t_{2}} + x_{0} \right) \\ &- \left( \frac{1}{2}a{t_{1}}^{2} + v_{0}{t_{1}} + x_{0} \right) \\
  &= \frac{1}{2}a\left( {t_{2}}^{2} - {t_{1}}^{2} \right) + v_{0}\left( {t_{2}} - {t_{1}} \right) \\
  &= \frac{1}{2}a\left( {t_{2}} - {t_{1}} \right)\left( {t_{2}} + {t_{1}} \right) + v_{0}\left( {t_{2}} - {t_{1}} \right) \\
  &= \boxed{\frac{1}{2}a\left( \Delta t \right)\left( {t_{2}} + {t_{1}} \right) + v_{0}\left( \Delta t \right)}
\end{align}

\divider

\prob{Difference equations}
\statement{To implement a difference equation version, calculate the initial versions of $x_{i}$, $v_{i}$, and $t_{i}$ based on $x_{0}$, $v_{0}$, and $t_{0}$. Then calculate the \textit{next} versions $x_{i + 1}$, $v_{i + 1}$, and $t_{i + 1}$ based on the \textit{current} versions $x_{i}$, $v_{i}$, and $t_{i}$ and previous time $t_{i - 1}$.}

\begin{align}
x_{i} &= x_{0} + v_{0} t_{0} + \frac{1}{2} a {t_{0}}^{2} \\
v_{i} &= v_{0} + a t_{0} \\
t_{i} &= t_{0}
\end{align}
For $\Delta t = \left( {t_{i}} - {t_{i - 1}} \right)$...
\begin{align}
x_{i + 1} &= x_{i} + v_{0} \left( \Delta t \right) + \frac{1}{2} a \left( \Delta t \right) \left( t_{i} + t_{i - 1} \right) \label{x-i1} \\
v_{i + 1} &= v_{i} + a \left( \Delta t \right) \\
t_{i + 1} &= t_{i} + \left( \Delta t \right)
\end{align}

\divider

\prob{Code}
\statement{The following \href{https://docs.python.org/3/}{Python} code (\texttt{accel.py}) demonstrates the instantaneous (continuous, differential) and recursive (discrete, difference) versions. They yield the same results.}

\begin{lstlisting}
#!/usr/bin/env python3
#
# accel.py
#

# initial values
x0, v0, t0, dt, a, tot = 3, 5, 7, 2, 10, 20

# instantaneous
for t in range(t0, tot + t0 + dt, dt):
    # print('*', t, x0, v0 * t, a * t * t / 2)            # debug print
    x = x0 + v0 * t + a * t * t / 2
    v = v0 + a * t
    print(f"{x:6.1f} {v:6.1f} {t:6.1f}")

print()

# recursive
xi, vi, ti = x0 + v0 * t0 + a * t0 * t0 / 2, v0 + a * t0, t0
print(f"{xi:6.1f} {vi:6.1f} {ti:6.1f}")
for t in range(t0 + dt, tot + t0 + dt, dt):
    # print('*', ti, xi, v0 * dt, a * dt * (t + ti) / 2)  # debug print
    # xi = xi + v0 * dt + a * dt * dt / 2                 # *not* dt ** 2
    # xi = xi + v0 * dt + a * (t * t - ti * ti) / 2       # alternate version
    xi = xi + v0 * dt + a * dt * (t + ti) / 2
    vi = vi + a * dt
    ti = ti + dt
    print(f"{xi:6.1f} {vi:6.1f} {ti:6.1f}")

\end{lstlisting}

Which results (with these example initial values) in:

\begin{lstlisting}
 283.0   75.0    7.0
 453.0   95.0    9.0
 663.0  115.0   11.0
 913.0  135.0   13.0
1203.0  155.0   15.0
1533.0  175.0   17.0
1903.0  195.0   19.0
2313.0  215.0   21.0
2763.0  235.0   23.0
3253.0  255.0   25.0
3783.0  275.0   27.0

 283.0   75.0    7.0
 453.0   95.0    9.0
 663.0  115.0   11.0
 913.0  135.0   13.0
1203.0  155.0   15.0
1533.0  175.0   17.0
1903.0  195.0   19.0
2313.0  215.0   21.0
2763.0  235.0   23.0
3253.0  255.0   25.0
3783.0  275.0   27.0
>>> 
\end{lstlisting}

The code \texttt{xi = xi + v0 * dt + a * dt * dt / 2} (that uses ${\left( \Delta t \right)}^{2}$ and is commented out) is not correct. The discrete calculation must include the last term of equation \eqref{x-i1} to match the continuous calculation.

If $a(t) \ne \text{constant}$, then this analysis does not apply .

\end{multicols*}
\end{document}
