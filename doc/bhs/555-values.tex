\documentclass[10pt]{article}%
%%%%%%%%%%%%%%%%%%%%%%%%%%%%%%%%%%% Packages
\input ../../input/input-packages.tex

%%%%%%%%%%%%%%%%%%%%%%%%%%%%%%%%%%% Document data
\input ../../input/input-data.tex

\def\dochomework{}
\def\hw{555 Values} % Change document homework header
\def\docdate{2019/10/01} % Change document date
\def\docphone{+1.772.444.2710} % Change document phone

%%%%%%%%%%%%%%%%%%%%%%%%%%%%%% Macros
\input ../../input/input-macros.tex

%%%%%%%%%%%%%%%%%%%%%%%%%%%%%% Headers & footers
\input ../../input/input-header-footer.tex

%%%%%%%%%%%%%%%%%%%%%%%%%%%%%% Document style
\input ../../input/input-style.tex

%%%%%%%%%%%%%%%%%%%%%%%%%%%%%% Table of contents / index / bibliography
\input ../../input/input-toc-idx-bib.tex

%%%%%%%%%%%%%%%%%%%%%%%%%%%%%% Points and solutions
\input ../../input/exam-points-solutions.tex
%\printanswers% to print solutions

\begin{document}
\setlength{\baselineskip}{1.2\baselineskip}%
\begin{multicols*}{2}

%%%%%%%%%%%%%%%%%%%%%%%%%%%%%%%%%%%%%%%%%%%%%%%%%%%%%%%%%%%%%%%%%%%%%%
\prob{Formulas}
\statement{The following formulas for resistor values in the astable comfiguration of a 555 timer are derived from one of many \href{https://circuitdigest.com/calculators/555-timer-astable-circuit-calculator}{on-line calculators} and are based on four parameters: \begin{itemize} \item $N$, the base note of the scale $[Hz]$; \item $D$, the degree of the scale on $[0, 12]$; \item $C_{1}$, the capacitance $[F]$; and \item $d$, the duty cycle of the square wave ($> \frac{1}{2}$).\end{itemize}}
\begin{align}
 T_{H} &= \ln(2) \times \left( R_{1} + R_{2} \right) \times C_{1} \\
 T_{L} &= \ln(2) \times \left( R_{2} \right) \times C_{1} \\
 T &= \frac{1}{\left( N \times {2}^{\frac{D}{12}} \right)} \\
  &= T_{H} + T_{L} \\
  &= \ln(2) \times \left( R_{1} + 2R_{2} \right) \times C_{1} \\
 \therefore \left( R_{1} + 2R_{2} \right) &= \boxed{\frac{1}{\left( \ln(2) \times \left( N \times {2}^{\frac{D}{12}} \right) \times C_{1} \right)}} \\
  \addlinespace\midrule\addlinespace
 d &= \frac{T_{H}}{T_{H} + T_{L}} = \frac{\left( R_{1} + R_{2} \right)}{\left( R_{1} + 2R_{2} \right)} > \frac{1}{2}\\
 d \left( R_{1} + 2R_{2} \right) &= \left( R_{1} + R_{2} \right) \\
 d R_{1} + 2d R_{2} &= R_{1} + R_{2} \\
 2d R_{2} - R_{2} &= R_{1} - d R_{1} \\
 \left( 2d - 1 \right) R_{2} &= \left( 1 - d \right) R_{1} \\
 \therefore R_{2} &= \boxed{\frac{\left( 1 - d \right)}{\left( 2d - 1 \right)} R_{1}} \\
  \addlinespace\midrule\addlinespace
\left( R_{1} + 2R_{2} \right) &= R_{1} + 2 \frac{\left( 1 - d \right)}{\left( 2d - 1 \right)} R_{1} \\
  &= \frac{{\left( 2d - 1 \right)}}{{\left( 2d - 1 \right)}} R_{1} + \frac{\left( 2 - 2d \right)}{\left( 2d - 1 \right)} R_{1} \\
  &= \frac{1}{{\left( 2d - 1 \right)}} R_{1} \\
 \therefore R_{1} &= \left( 2d - 1 \right) \left( R_{1} + 2R_{2} \right) \\
  &= \boxed{\frac{\left( 2d - 1 \right)}{\left( \ln(2) \times \left( N \times {2}^{\frac{D}{12}} \right) \times C_{1} \right)}} \Omega
\end{align}

\divider

\end{multicols*}
\end{document}
