\documentclass[11pt]{article}%
%%%%%%%%%%%%%%%%%%%%%%%%%%%%%%%%%%% Packages
\input ../input/input-packages.tex

%%%%%%%%%%%%%%%%%%%%%%%%%%%%%%%%%%% Document data
\input ../input/input-data.tex

\def\docmassbaypaper{}
\def\docname{This is the title}
\def\docdate{2021/07/16}
\def\docorg{MassBay Consulting}
\def\docemail{dcp@massbay.consulting}
\def\docphone{+1.772.444.2710}
\def\docversion{0.1} % Change document version

%%%%%%%%%%%%%%%%%%%%%%%%%%%%%% Macros
\input ../input/input-macros.tex

%%%%%%%%%%%%%%%%%%%%%%%%%%%%%% Headers & footers
\input ../input/input-header-footer.tex

%%%%%%%%%%%%%%%%%%%%%%%%%%%%%% Document style
\input ../input/input-style.tex

%%%%%%%%%%%%%%%%%%%%%%%%%%%%%% Table of contents / index / bibliography
\input ../input/input-toc-idx-bib.tex
\addbibresource{../bib/dcpbiblatex.bib}

%%%%%%%%%%%%%%%%%%%%%%%%%%%%%% Points and solutions
\input ../input/exam-points-solutions.tex
%\printanswers% to print solutions

%%%%%%%%%%%%%%%%%%%%%%%%%%%%%% Title
\title{\textbf{\docname} \\\vskip \baselineskip \href{https://massbay.consulting/}{\includegraphics*[height = 0.75in]{../images/MassBayLogo1085x337.pdf}}}%

%%%%%%%%%%%%%%%%%%%%%%%%%%%%%% Author(s) (including \thanks)
\def\docauthortwo{Clarence T. Fishburn}
\def\docorgtwo{Simitongue Institute}
\def\docemailtwo{clarence.fishburn@gmail.com}

\fancyfoot[L]{\scriptsize\bf\begin{tabular}[t]{c@{\extracolsep{2em}}c} \docauthortwo & \docauthor \\ \lemail{\docemailtwo} & \lemail{\docemail}\end{tabular}}%

% $^\dagger$% \thanks already uses \textasteriskcentered \textdagger \textdaggerdbl \textsection \textparagraph \textbardbl \textasteriskcentered\textasteriskcentered \textdagger\textdagger \dagger\dagger \textdaggerdbl\textdaggerdbl
\author{\docauthortwo \thanks{The author wishes to thank poop.} \\ \docorgtwo \\ \small{\lemail{\docemailtwo}}
 \and% Use \and to add author information
 \docauthor \thanks{The authors also wish to thank their families.} \\ \docorg \\ \small{\lemail{\docemail}}}%

%%%%%%%%%%%%%%%%%%%%%%%%%%%%%% Date
\date{} %

%%%%%%%%%%%%%%%%%%%%%%%%%%%%%% Document
\begin{document}

\maketitle

\begin{multicols*}{2}
\begin{abstract}

%%%%%%%%%%%%%%%%%%%%%%%%%%%%%% Keywords & Abstract

{\em Keywords: keyword1, keyword2, keyword3 %

In this paper there are examples of all the useful things you can do
with \LaTeX. This document includes settings and examples for using
the following packages: {\tt amsthm, appendix, array, booktabs, cancel, ccicons, comment, empheq, endnotes, enumitem, float, graphicx, indentfirst, lastpage, lipsum, listings, makeidx, mathtools, multicol, multirow, palatino, paracol, setspace, standalone, tabularray, tabularx, verbatim, wrapfig, hyperref, geometry, xolor, tcolorbox, textpos, tikz, tkz-euclide, titlesec, eucal, mathdesign, inconsolata}.

}

\end{abstract}

%%%%%%%%%%%%%%%%%%%%%%%%%%%%%% Table of Contents
%\tableofcontents %

\baselineskip = 1.25\baselineskip % paragraph spacing

%%%%%%%%%%%%%%%%%%%%%%%%%%%%%% Introduction
\section{Introduction}
\label{Introduction}

This paper is a template for writing \LaTeX papers for
\href{http://massbay.net}{MassBay Consulting, LLC} and other
audiences.

In this paper there are examples of all the useful things you can do
with \LaTeX. This document includes settings and examples for using
the following packages: {\tt amsthm, appendix, array, booktabs, cancel, ccicons, comment, empheq, endnotes, enumitem, float, graphicx, indentfirst, lastpage, lipsum, listings, makeidx, mathtools, multicol, multirow, palatino, paracol, setspace, standalone, tabularray, tabularx, verbatim, wrapfig, hyperref, geometry, xolor, tcolorbox, textpos, tikz, tkz-euclide, titlesec, eucal, mathdesign, inconsolata}.

\subsection{The quick brown fox}
\label{The quick brown fox}

The quick\index{quick} brown\index{brown} fox\index{fox}
jumps\index{jumps} over\index{over} the lazy\index{lazy}
dog\index{dog}. The quick brown fox jumps over the lazy dog. The
quick brown fox jumps over the lazy dog. The quick brown fox jumps
over the lazy dog. The quick brown fox jumps over the lazy dog. The
quick brown fox jumps over the lazy dog. The quick brown fox jumps
over the lazy dog. The quick brown fox jumps over the lazy dog. The
quick brown fox jumps over the lazy dog.

%%%%%%%%%%%%%%%%%%%%%%%%%%%%%% This Section
\section{This Section}
\label{This Section}

In section \ref{This Section} there are examples of the various
types of lists\footnote{These correspond to the {\tt \textless
dl\textgreater}, {\tt \textless ol\textgreater}, \& {\tt \textless
ul\textgreater}\ HTML tags.}:

\begin{description}
\item[Item 1 header] This is item one\dots The quick brown fox jumps
over the lazy dog.
\item[Item 2 header] This is item two\dots The quick brown fox jumps
over the lazy dog.
\item[Item 3 header] This is item three\dots The quick brown fox jumps
over the lazy dog.
\end{description}

\begin{enumerate}
\item {\bf Item 1 header} --- This is item one\dots The quick brown fox jumps
over the lazy dog.
\item {\bf Item 2 header} --- This is item two\dots The quick brown fox jumps
over the lazy dog.
\item {\bf Item 3 header} --- This is item three\dots The quick brown fox jumps
over the lazy dog.
\end{enumerate}

\begin{itemize}
\item {\bf Item 1 header} --- This is item one\dots The quick brown fox jumps
over the lazy dog.
\item {\bf Item 2 header} --- This is item two\dots The quick brown fox jumps
over the lazy dog.
\item {\bf Item 3 header} --- This is item three\dots The quick brown fox jumps
over the lazy dog.
\end{itemize}

%%%%%%%%%%%%%%%%%%%%%%%%%%%%%% The Other Section
\section{The Other Section}
\label{The Other Section}

In section \ref{The Other Section} there are examples of
mathematics:\begin{align}
y &= mx+b && \text{Linear} \\
f(x) &= \int_{-\infty}^{+\infty} e^{i\theta} d\theta && \text{Integral} \\
z &= 2^k-\binom{k}{1}2^{k-1}+\binom{k}{2}2^{k-2} &&\text{Binomial}
\end{align}

%%%%%%%%%%%%%%%%%%%%%%%%%%%%%% The Code Section
\section{The Code Section}
\label{The Code Section}

\begin{lstlisting}[language=Python,caption=\code{m4tacolor.py} module,label=m4tacolor]
# 345678901234567890123456789012345678901234567890123456789012345678901234567890
#!/usr/local/env python3
def gcd(m, n):
    """Return GCD of m and n. GCD is defined for all integers."""
    return abs(m) if n == 0 else gcd(n, m % n)          # abs allows any integer

# https://en.wikipedia.org/wiki/Least_common_multiple
def lcm(m, n):
    """Return LCM of m and n. LCM is defined for all integers."""
    return 0 if m * n == 0 else abs(m // gcd(m, n) * n) # abs allows any integer
\end{lstlisting}

%%%%%%%%%%%%%%%%%%%%%%%%%%%%%% Conclusion
\section{Conclusion}
\label{Conclusion}

This last section provides a citation so that a bibliography is
automatically included \cite{book:knr}, \cite{book:middlemarch}, \&
\cite{book:engines-of-logic}.

%%%%%%%%%%%%%%%%%%%%%%%%%%%%%% Bibliography
\printbibliography % To update, delete *.bbl

%%%%%%%%%%%%%%%%%%%%%%%%%%%%%% Index
%\printindex % To update, delete *.ind

\end{multicols*}
\end{document}
