\documentclass[11pt]{article}%
%%%%%%%%%%%%%%%%%%%%%%%%%%%%%%%%%%% Packages
\input ../input/input-packages.tex

%%%%%%%%%%%%%%%%%%%%%%%%%%%%%%%%%%% Document data
\input ../input/input-data.tex

\def\docdcppaper{}
\def\doctitle{This is the title}
\def\docdate{2021/07/16}
\def\docemail{dcp@mail.com}
\def\docphone{+1.772.444.2710}
\def\docversion{0.1} % Change document version

%%%%%%%%%%%%%%%%%%%%%%%%%%%%%% Macros
\input ../input/input-macros.tex

%%%%%%%%%%%%%%%%%%%%%%%%%%%%%% Headers & footers
\input ../input/input-header-footer.tex

%%%%%%%%%%%%%%%%%%%%%%%%%%%%%% Document style
\input ../input/input-style.tex

\lstset{% general command to set parameter(s) for listings
  basicstyle=\singlespacing\ttfamily\fontsize{5}{6}\selectfont,
  numberstyle=\fontsize{5}{6}\selectfont,
}

%%%%%%%%%%%%%%%%%%%%%%%%%%%%%% Table of contents / index / bibliography
\input ../input/input-toc-idx-bib.tex

%%%%%%%%%%%%%%%%%%%%%%%%%%%%%% Points and solutions
\input ../input/exam-points-solutions.tex
%\printanswers% to print solutions

%%%%%%%%%%%%%%%%%%%%%%%%%%%%%% Title
\title{{\bf \doctitle}}%

%%%%%%%%%%%%%%%%%%%%%%%%%%%%%% Author(s) (including \thanks)
\def\docauthortwo{Clarence T. Fishburn}
\def\docschooltwo{Simitongue Institute}
\def\docemailtwo{clarence.fishburn@gmail.com}

\fancyfoot[L]{\scriptsize\bf\begin{tabular}[t]{c@{\extracolsep{2em}}c} \docauthortwo & \docauthor \\ \lemail{\docemailtwo} & \lemail{\docemail}\end{tabular}}%

\author{\docauthortwo \\ \docschooltwo \\ \small{\lemail{\docemailtwo}}
 \and \docauthor$^\dagger$%
  \thanks{$\dagger$ The authors wish to thank their families.} \\ \docschool \\ \small{\lemail{\docemail}}}%

%%%%%%%%%%%%%%%%%%%%%%%%%%%%%% Date
\date{} %

%%%%%%%%%%%%%%%%%%%%%%%%%%%%%% Dcoument
\begin{document}

\maketitle

\begin{multicols*}{2}
\begin{abstract}

%%%%%%%%%%%%%%%%%%%%%%%%%%%%%% Keywords & Abstract

{\em Keywords: keyword1, keyword2, keyword3 %

In this paper there are examples of all the useful things you can do
with \LaTeX. This document includes settings and examples for using
the following packages: {\tt amsmath, amsthm, comment, enumitem,
fancyhdr, geometry, graphicx, hyperref, makeidx, natbib, palatino,
titlesec, totcbibind, \& verbatim} plus the font packages: {\tt
amsfonts,amssymb, cmmib57, eucal, (eufrak, euscript), \& latexsym}.

}

\end{abstract}

%%%%%%%%%%%%%%%%%%%%%%%%%%%%%% Table of Contents
%\tableofcontents %

\baselineskip = 1.25\baselineskip % paragraph spacing

%%%%%%%%%%%%%%%%%%%%%%%%%%%%%% Introduction
\section{Introduction}
\label{Introduction}

This paper is a template for writing \LaTeX\ papers for
\href{http://j.mp/psb_david_petty}{Brookline High School} and other
audiences.

%\begin{comment} %%%%%%%%%% COMMMENT

In this paper there are examples of all the useful things you can do
with \LaTeX. This document includes settings and examples for using
the following packages: {\tt amsmath, amsthm, comment, enumitem,
fancyhdr, geometry, graphicx, hyperref, makeidx, natbib, palatino,
titlesec, totcbibind, \& verbatim} plus the font packages: {\tt
amsfonts,amssymb, cmmib57, eucal, (eufrak, euscript), \& latexsym}

\lipsum[1]

\subsection{The quick brown fox}
\label{The quick brown fox}

The quick\index{quick} brown\index{brown} fox\index{fox}
jumps\index{jumps} over\index{over} the lazy\index{lazy}
dog\index{dog}. \lipsum[2]

%%%%%%%%%%%%%%%%%%%%%%%%%%%%%% This Section
\section{This Section}
\label{This Section}

In section \ref{This Section} there are examples of the various
types of lists\footnote{These correspond to the {\tt \textless
dl\textgreater}, {\tt \textless ol\textgreater}, \& {\tt \textless
ul\textgreater}\ HTML tags.}:

\begin{description}
\item[Item 1 header] This is item one\dots The quick brown fox jumps
over the lazy dog.
\item[Item 2 header] This is item one\dots The quick brown fox jumps
over the lazy dog.
\item[Item 3 header] This is item one\dots The quick brown fox jumps
over the lazy dog.
\end{description}

\begin{enumerate}
\item {\bf Item 1 header} --- This is item one\dots The quick brown fox jumps
over the lazy dog.
\item {\bf Item 2 header} --- This is item two\dots The quick brown fox jumps
over the lazy dog.
\item {\bf Item 3 header} --- This is item three\dots The quick brown fox jumps
over the lazy dog.
\end{enumerate}

\begin{itemize}
\item {\bf Item 1 header} --- This is item one\dots The quick brown fox jumps
over the lazy dog.
\item {\bf Item 2 header} --- This is item two\dots The quick brown fox jumps
over the lazy dog.
\item {\bf Item 3 header} --- This is item three\dots The quick brown fox jumps
over the lazy dog.
\end{itemize}

%%%%%%%%%%%%%%%%%%%%%%%%%%%%%% The Other Section
\section{The Other Section}
\label{The Other Section}

In section \ref{The Other Section} there are examples of
mathematics:\begin{align}
y &= mx+b && \text{Linear} \\
f(x) &= \int_{-\infty}^{+\infty} e^{i\theta} d\theta && \text{Integral} \\
z &= 2^k-\binom{k}{1}2^{k-1}+\binom{k}{2}2^{k-2} &&\text{Binomial}
\end{align}

%%%%%%%%%%%%%%%%%%%%%%%%%%%%%% The Code Section
\section{The Code Section}
\label{The Code Section}

\begin{lstlisting}[language=Python,caption=\code{m4tacolor.py} module,label=m4tacolor]
# 345678901234567890123456789012345678901234567890123456789012345678901234567890
#!/usr/local/env python3
#
# m4tacolor.py
#
# How many mxm patterns are there on an nxn chess board n >=m?
#
def getPattern(i, m, n, x):
    """Return the ith mxm pattern in a nxn board represented by x."""
    assert n >= m, 'n < m'
    pattern, gap, mask = 0, n - m, 2 ** m - 1
    for j in range(m):
        pattern += x >> i + j * gap & mask << j * m
        #print('*', m, n, i, j, gap, i + j * gap, 
        #    x >> i + j * gap, mask, mask << j * m, pattern)
    return pattern

def getPatterns(m, n, x):
    """Return the set of mxm patterns in a nxn board represented by x."""
    assert n >= m, 'n < m'
    patterns = set()
    for i in range(n - m + 1):
        for j in range(n - m + 1):
            patterns.add(getPattern(i + j * n, m, n, x))
    return patterns

def square(k, x):
    """Return the unicode black or white square based on the kth bit of x."""
    #return '\u25a0' if 1 << k & x else '\u25a1'
    return 'X' if 1 << k & x else 'O'

def board(x, n):
    """Return the string representation of x as an nxn board."""
    return '\n'.join([ ''.join(
        [ square(i + j * n, x) for i in range(n) ]) for j in range(n) ])

tests = [ (2, 5,), (4, 259,), (5, 262149,) ]
for m, n in tests:
    f = 'm={} n={}'
    print(f.format(m, n))
    f += ' board={:0' + str(n * n) + 'b} {}=len({})'
    for x in range(1000, 2 ** (n * n)):
        patterns = getPatterns(m, n, x)
        if len(patterns) == 2 ** (m * m):
            print(f.format(m, n, x, len(patterns), patterns))
            print(board(x, n))
            break
\end{lstlisting}

%%%%%%%%%%%%%%%%%%%%%%%%%%%%%% Conclusion
\section{Conclusion}
\label{Conclusion}

This last section provides a citation so that a bibliography is
automatically included for \citet{book:linear-algebra} \& \citet{book:middlemarch}.

%\end{comment} %%%%%%%%%% COMMMENT

%%%%%%%%%%%%%%%%%%%%%%%%%%%%%% Bibliography
\bibliography{../templates/bib/dcpbibtex} % To update, delete *.bbl

%%%%%%%%%%%%%%%%%%%%%%%%%%%%%% Index
%\printindex % To update, delete *.ind

\end{multicols*}
\end{document}
