\documentclass[12pt]{article}%
\usepackage{amsmath,amsthm,comment,enumitem,fancyhdr,graphicx,hyperref,makeidx,multicol,verbatim}%
\usepackage[psamsfonts]{amsfonts,amssymb,eucal}% ftp://ftp.ams.org/pub/tex/doc/amsfonts/amsfndoc.pdf
\usepackage{cmmib57,latexsym}%
\usepackage[paper=letterpaper, top=0.25in, bottom=0.5in, left=0.75in, right=0.75in, includeheadfoot, head=0.5in]{geometry}
\usepackage[small,compact]{titlesec}%

%%%%%%%%%%%%%%%%%%%%%%%%%%%%%%%%%%% Document data
\def\docclass{328-DP} % Change document class
\def\docdate{2005-2006 Q1} % Change document date
\def\docname{Quiz 1} % Change document name
\def\docnumber{1234567} % Change document number

%%%%%%%%%%%%%%%%%%%%%%%%%%%%%% Macros
\def\foreign{\em}
\def\etc{{\foreign etc.}}
\def\eg{{\foreign e.g.}}

%%%%%%%%%%%%%%%%%%%%%%%%%%%%%% Headers & footers data
\fancyhf{} % clear all header and footer fields
\fancyhead[L]{\bf \baselineskip = 1.5\baselineskip \docdate\\\docclass}%
\fancyhead[C]{\bf \baselineskip = 1.5\baselineskip \docname\\\docnumber}%
\fancyhead[R]{\bf \baselineskip = 1.5\baselineskip Name: \vbox{\hrule width 2in}\\
Date: \vbox{\hrule width 2in}}%
\fancyfoot[L]{\scriptsize \bf David C. Petty\\POBox 400222
\textbullet \ Cambridge MA 02140-0003 USA \textbullet \
\texttt{\textless dcp@mail.com\textgreater} \textbullet \
+1.617.331.2345}%
\fancyfoot[C]{\scriptsize \bf }%
\fancyfoot[R]{\scriptsize \bf \ \thepage\\\jobname.pdf}%
\renewcommand{\headrulewidth}{0.5pt}
\renewcommand{\footrulewidth}{0.5pt}

%%%%%%%%%%%%%%%%%%%%%%%%%%%%%% Document style
\setlength{\columnsep}{1cm}%
\setlength{\parindent}{0cm}%
\setlength{\parskip}{\baselineskip}%
\setlist{nolistsep, itemsep=.125\baselineskip}%
\setenumerate{leftmargin=*, label={\arabic*)}}%
\setdescription{style=multiline, leftmargin=3cm}%
\setitemize{leftmargin=*, label={\tiny$\blacktriangleright$}}%
%\numberwithin{equation}{section}%
\pagestyle{fancy}%

%%%%%%%%%%%%%%%%%%%%%%%%%%%%%% Document
\begin{document}

Isn't \TeX \ cool? This document shows how you can use the
typographic system \TeX \ to write stuff like: If $\rho$ and
$\theta$ are both positive, then $f(\theta) -{\mit \Gamma}_{\theta}
< f(\rho)-{\mit \Gamma}_{\rho}$.

To write a formula, you use backslash control sequences (like {\tt
$\backslash$root} and {\tt $\backslash$frac}) and various types of
braces (like {\tt \{\}}) to create formulas that are then typeset
beautifully. There are many quickie guides on the net', {\em e.g.}
\url{http://www-math.mit.edu/18.821/short-math-guide.pdf} \&
\url{ftp://ftp.ams.org/pub/tex/doc/amsmath/amsldoc.pdf} \&
\url{http://www.bitjungle.com/~isoent/isoent-ref.pdf}.

Here's another example:
\begin{align}
 r &= \frac{-b \pm \root \of {b^2-4ac}}{2a} &\text{The Quadratic Formula}
\end{align}
To typeset the quadratic formula (above), use this `code:'
\begin{verbatim}
 r &= \frac{-b \pm \root \of {b^2-4ac}}{2a} &\text{The Quadratic Formula}
\end{verbatim}
To understand how that `code' results in the typeset formula you
need to know that the {\tt $\backslash$frac} control sequence takes
two arguments, {\tt $\backslash$pm} inserts plus-or-minus, the `{\tt
\&}' characters tell \TeX \ where to line up the columns, {\tt
$\backslash$of} is part of {\tt $\backslash$root}, {\em etc.}

Here are some more examples:
\begin{align}
y &= \left[
  \begin{array}{ c c c }
    x_{00} & x_{01} & x_{02} \\
    x_{10} & x_{11} & x_{12} \\
    x_{20} & x_{21} & x_{22}
  \end{array} \right]
  & \text{This is an example of an array with subscripts.}
\end{align}

And this ia an example of `{\em Solve for x}:'
\begin{align*}
 x + 3 &= 2x - 4 \\
 x + 3 + 4 &= 2x - 4 + 4 &&\text{Add 4 to both sides.} \\
 x + 7 &= 2x &&\text{Combine like terms.} \\
 x + 7 - x &= 2x - x &&\text{Subtract x from both sides.} \\
 7 &= x &&\text{Combine like terms.}
\intertext{Check your results\dots}
\left( 7 \right) + 3 &= 2\left( 7 \right) - 4 &&\text{Substitute 7 for x.} \\
 10 &= 14 - 4 &&\text{Simplify.} \\
 10 &= 10 &&\text{\checkmark}
 \end{align*}

%\begin{align*}A&B\\M&N\end{align*}
%\begin{align*}A&B&C\\M&N&O\end{align*}
%\begin{align*}A&B&C&D\\M&N&O&P\end{align*}
%\begin{align*}A&B&C&D&E\\M&N&O&P&Q\end{align*}
%\begin{align*}A&B&C&D&E&F\\M&N&O&P&Q&R\end{align*}

\end{document}
